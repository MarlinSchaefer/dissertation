\chapter{Conclusions and Outlook}\label{ch:conclusions}
This thesis has analyzed the applicability of \acrshort{ml} methods to the problem of detecting \acrshort{gw} signals from \acrshort{cbc} sources in comparatively strong noise. It studied both \acrshort{bns} and \acrshort{bbh} signals and introduced novel methods to expand the capabilities of existing \acrshort{ml} algorithms. A core contribution of this thesis is an objective comparison of sensitivities between \acrshort{ml} methods and existing search pipelines. Several studies highlighted the importance of calculating sensitivities normalized by the population of \acrshort{gw} sources derived from long-duration continuous data to make representative claims about \acrshort{gw} detection capability.

Each chapter of this thesis has solved select problems of \acrshort{ml} based searches for \acrshort{cbc} signals to advance the field from a proof of principle level to actual applications. Chapter \ref{ch:bns} introduced a novel deep learning based search for \acrshort{bns} signals that outperforms previous \acrshort{ml} methods at low \acrshort{far}s. While it is not yet competitive with matched filter searches, it provides a method to reduce the number of data samples that need to be processed. This reduction in data size for \acrshort{bns} signals is crucial for \acrshort{nn}s, as they struggle with large inputs. Chapter \ref{ch:training_strats} introduced a simple modification to the \acrshort{nn}s used in early deep learning \acrshort{bbh} search algorithms. This extension enabled the algorithm to operate at \acrshort{far}s $\mathcal{O}(1)$ per month and be competitive to a matched filter baseline. Chapter \ref{ch:cnn_coinc} applied a coincidence strategy to deep learning searches optimized on a single detector and showed that the background can be trivially extended to test the algorithm down to \acrshort{far}s $\mathcal{O}(10^{-2})$ per year. Chapter \ref{ch:mlgwsc1} presents the results of a global mock data challenge organized by the author of this thesis. The mock data challenge makes the tools and experiences gained from chapters \ref{ch:bns} to \ref{ch:cnn_coinc} publicly available by publishing open source software and reference data sets that can be used to evaluate any search on the provided parameter space. The evaluation results can be compared to existing literature, as the study includes reference sensitivities from current production search pipelines.

The summarized takeaways from the various analyses presented in this thesis are as follows. \acrshort{nn}s are already competitive in sensitivity to matched filtering when \acrshort{bbh} signals are considered. This is highlighted by the results found in chapter \ref{ch:mlgwsc1}. There, the best \acrshort{ml} search retains $\geq 70\%$ of the sensitivity achieved by the state-of-the-art matched filter based PyCBC search pipeline and becomes comparable at \acrshort{far}s $> 100$ per month, even in real noise. However, while the most sensitive algorithm requires less time to process the data than PyCBC, the probed parameter region is still efficiently searched by existing methods. In regions of parameter space, where matched filtering becomes computationally more expensive, deep learning is also limited in its capability. Especially long duration \acrshort{bns} and \acrshort{nsbh} signals are still challenging to \acrshort{nn} searches. This is in line with chapter \ref{ch:bns} where we observed that, while we improve the state-of-the-art for deep learning \acrshort{bns} searches, our algorithm is handily outperformed by matched filtering. We also found evidence of the same problem in chapter \ref{ch:mlgwsc1}, where even high \acrshort{snr} long duration signals were missed by all deep learning searches. Furthermore, the results presented in chapter \ref{ch:cnn_coinc} suggest that a major problem for \acrshort{ml} algorithms in the future may be the unavailability of signal consistency tests to reject many noise artifacts. Typically deep learning searches are used as binary classifiers that decide only between the presence and absence of a signal in the parameter region they were trained on. The lack of crude source parameter estimates in this setting complicates a coincidence analysis, which greatly decreases the \acrshort{far}s that can be trivially tested.

Challenges for \acrshort{ml} based \acrshort{gw} search algorithms are plentiful, but should not be discouraging. Great progress has been made in the last few years in terms of capability and clarity of results. Their enormous potential, the rapid development of deep learning methods that can be imported from other research areas, and the easy utilization of graphics cards are all good reasons to continue trying to overcome existing problems. Furthermore, some \acrshort{ml} algorithms are already invaluable tools in some areas of \acrshort{gw} astronomy. These include glitch classification~\cite{Zevin:2016qwy} and improvements to ranking statistics to reject non-Gaussian noise artifacts~\cite{Mishra:2021tmu}.

The identified weaknesses of current approaches also provide a clear path for future research. It is necessary to develop \acrshort{ml} algorithms capable of reliably and rapidly detecting weak, long duration \acrshort{gw}s from sources such as \acrshort{bns} or \acrshort{nsbh} mergers. These sources are of special interest as they potentially emit an \acrshort{em} counterpart, which can be used to extract further information from the system and constrain stellar models. Their rapid detection can increase the \acrshort{em} observation time, hopefully to a point where the prompt emission can be observed. It is also desirable to develop \acrshort{ml} \acrshort{gw} searches that operate at low \acrshort{far}s on the order of one per year. A promising route studied in this thesis is the adoption of a coincidence analysis scheme as it is used by state-of-the-art production searches. To achieve this goal and reliably reject non-Gaussian noise artifacts, developing signal consistency checks for deep learning based algorithms is important. Organizing future mock data challenges would be beneficial to benchmark the progress of the field accurately. The mock data challenge discussed in chapter \ref{ch:mlgwsc1} has provided a baseline that can be easily extended to more difficult regions of parameter space.

Beyond the clear path painted by the challenges of existing methods identified in this thesis, there are further interesting avenues to apply \acrshort{ml} in \acrshort{gw} astronomy. One approach to utilize \acrshort{ml} in \acrshort{gw} detection that has often been proposed but never actually been explored beyond an initial study~\cite{Verma:2021epx} is to do a hierarchical search. Fast \acrshort{ml} algorithms can be used to flag candidate detections at high \acrshort{far}s, thereby rejecting most of the noise. The candidate detections can then be checked using matched filtering to reduce computational costs while preserving the sensitivity of state-of-the-art analyses. The work in this thesis has already proven that deep learning searches can be more sensitive than matched filtering at very high \acrshort{far}s and may, therefore, be great first-stage filters. Another application not explored in this thesis that has gathered much interest in the recent past is the rapid production of posteriors to determine the parameters of \acrshort{gw} sources~\cite{Gabbard:2019rde, Dax:2021tsq}. Such development could reduce the computational costs by orders of magnitude due to the long runtimes of existing parameter estimation codes and the trivial evaluation of their deep learning counterparts.

Future \acrshort{gw} observation runs and detectors are expected to provide an increased rate of detections. The increased number of observations will allow us to make more reliable statements about the population of astrophysical objects and hopefully lead to new and exciting insights into the Universe. At the same time, the large number of expected detections provides a challenge for data analysts that need to process them in a timely manner. This is where \acrshort{ml} algorithms may be able to provide solutions previous algorithms cannot. How effective \acrshort{ml} algorithms will be applied in broad \acrshort{gw} astronomy tasks and how strongly they will be utilized remains to be seen. However, this thesis shows that \acrshort{ml} has already passed many of the hurdles needed for an application as a \acrshort{gw} search pipeline.
